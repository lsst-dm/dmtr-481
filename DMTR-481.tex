% generated from JIRA project LVV
% using template at /opt/hostedtoolcache/Python/3.11.14/x64/lib/python3.11/site-packages/docsteady/templates/tpr.latex.jinja2.
% using docsteady version 3.0.8
% Please do not edit -- update information in Jira instead
\documentclass[DM,lsstdraft,STR,toc]{lsstdoc}
\usepackage{geometry}
\usepackage{longtable,booktabs}
\usepackage{enumitem}
\usepackage{arydshln}
\usepackage{attachfile}
\usepackage{array}
\usepackage{dashrule}
\usepackage{pdfpages}

\newcolumntype{L}[1]{>{\raggedright\let\newline\\\arraybackslash\hspace{0pt}}p{#1}}

\input{meta.tex}

\newcommand{\attachmentsUrl}{https://github.com/\gitorg/\lsstDocType-\lsstDocNum/blob/\gitref/attachments}
\providecommand{\tightlist}{
  \setlength{\itemsep}{0pt}\setlength{\parskip}{0pt}}

\setcounter{tocdepth}{4}

\providecommand{\ul}[1]{\textbf{#1}}


% Pandoc >= 3.2.1 introduces \pandocbounded; fallback for older/custom templates
\providecommand{\pandocbounded}[1]{#1}

\begin{document}

\def\milestoneName{Science Pipelines Final Release}
\def\milestoneId{LDM-503-17a}
\def\product{Software Products}

\setDocCompact{true}

\title{LDM-503-17a: Science Pipelines Final Release Test Plan and Report}
\setDocRef{\lsstDocType-\lsstDocNum}
\date{ 2025-10-17 }
\author{ Yusra AlSayyad }

% Most recent last
\setDocChangeRecord{
\addtohist{}{2025-10-17}{First draft}{Yusra AlSayyad}
}

\setDocCurator{Yusra AlSayyad}
\setDocUpstreamLocation{\url{https://github.com/lsst-dm/\lsstDocType-\lsstDocNum}}
\setDocUpstreamVersion{\vcsRevision}



\setDocAbstract{
This is the test plan and report for
\textbf{ Science Pipelines Final Release} (LDM-503-17a),
an LSST milestone pertaining to the Data Management Subsystem.\\
This document is based on content automatically extracted from the Jira test database on \docDate.
The most recent change to the document repository was on \vcsDate.
}


\maketitle

\section{Introduction}
\label{sect:intro}


\subsection{Objectives}
\label{sect:objectives}

 This test plan checks for the successful release of ~LSST Science
Pipelines version 29.0.0. \hspace{0pt} Version 29.0.0 was the
\hspace{0pt}version used ~for Data \hspace{0pt}Preview 1 and was the
final release made during construction.~\\
\strut \\
It will demonstrate that:\\

\begin{itemize}
\tightlist
\item
  The release has been tagged, built and made available through standard
  distribution channels;
\item
  Release documentation, including release notes and a characterization
  report, are available on the Pipelines documentation website
  (https://pipelines.lsst.io/);
\item
  An end-user can follow standard instructions to install the release
  onto some representative system;
\item
  The release is installed into the ``shared stack'' on the rubin-devl
  shared developer systems at the Rubin Data Facility;
\item
  The pipelines\_check test package executes successfully in the context
  of the release.
\end{itemize}

This test plan does not, in itself, verify the scientific integrity or
algorithmic correctness of the release, beyond checking that defined
procedures for checking basic correctness and characterizing algorithmic
performance have been followed.\\
\strut \\

\subsection{Scope}\label{scope}

\hfill\break
\hfill\break
The overall strategy for testing and verification within Rubin Data
Management is described in \citeds{LDM-503}.\\
This test plan specifically verifies successful completion of milestone
LDM-503-16a, which refers to the Fall 2022 release of the LSST Science
Pipelines (and Spring 2023 verification of such).



\subsection{System Overview}
\label{sect:systemoverview}

 The LSST Science Pipelines comprise the scientific algorithms that will
be used to process data for the Rubin Observatory Legacy Survey of Space
and Time (Rubin\textquotesingle s LSST), ~arranged into executable
pipelines by means of the LSST ``task'' framework. They also include
execution middleware that is common across execution environments (for
example, the ``Data Butler'' I/O abstraction is included, but schedulers
or workflow management for specific clusters is not), and ``camera
packages'', which adapt and configure the algorithms for use with
specific instrumentation.\\
\strut \\

\subsection{\texorpdfstring{Applicable Documents\\
}{Applicable Documents }}\label{applicable-documents}

\hfill\break
\citeds{LDM-503} Data Management Test Plan\\
\citeds{LDM-151} Data Management Science Pipelines Design\\
\citeds{LSE-61} Data Management System Requirements


\subsection{Document Overview}
\label{sect:docoverview}

This document was generated from Jira, obtaining the relevant information from the
\href{https://rubinobs.atlassian.net/projects/LVV?selectedItem=com.atlassian.plugins.atlassian-connect-plugin:com.kanoah.test-manager__main-project-page\#\!/v2/testPlan/LVV-P83}{LVV-P83}
~Jira Test Plan and related Test Cycles (
\href{https://rubinobs.atlassian.net/projects/LVV?selectedItem=com.atlassian.plugins.atlassian-connect-plugin:com.kanoah.test-manager__main-project-page\#\!/testPlayer/LVV-R268}{LVV-R268}
).

Section \ref{sect:intro} provides an overview of the test campaign, the system under test (\product{}),
the applicable documentation, and explains how this document is organized.
Section \ref{sect:testplan} provides additional information about the test plan, like for example the configuration
used for this test or related documentation.
Section \ref{sect:personnel} describes the necessary roles and lists the individuals assigned to them.

Section \ref{sect:overview} provides a summary of the test results, including an overview in Table \ref{table:summary},
an overall assessment statement and suggestions for possible improvements.
Section \ref{sect:detailedtestresults} provides detailed results for each step in each test case.

The current status of test plan \href{https://rubinobs.atlassian.net/projects/LVV?selectedItem=com.atlassian.plugins.atlassian-connect-plugin:com.kanoah.test-manager__main-project-page\#\!/v2/testPlan/LVV-P83}{LVV-P83} in Jira is \textbf{ Completed }.

\subsection{References}
\label{sect:references}
\renewcommand{\refname}{}
\bibliography{lsst,refs,books,refs_ads,local}


\newpage
\section{Test Plan Details}
\label{sect:testplan}


\subsection{Data Collection}

  Observing is not required for this test campaign.

\subsection{Verification Environment}
\label{sect:hwconf}
  Several of the tests described in this plan are agnostic of environment:
they involve checking that certain content has been properly published.
This can be performed from any internet-connected system with a web
browser, and will, in this case, likely be executed from the
tester\textquotesingle s laptop.\\
\strut \\
Where tests require installation or execution of specific Science
Pipelines components, this will be carried out on the shared developer
infrastructure at the Rubin Data Facility. This infrastructure provides
a number of powerful systems accessible to Rubin developers.

  \subsection{Entry Criteria}
  None

  \subsection{Exit Criteria}
  None


\subsection{Related Documentation}

Docushare collection where additional relevant documentation can be found:

\begin{itemize}
\item None
\end{itemize}


\subsection{PMCS Activity}

Primavera milestones related to the test campaign:
None


\newpage
\section{Personnel}
\label{sect:personnel}

The personnel involved in the test campaign is shown in the following table.

{\small
\begin{longtable}{p{3cm}p{3cm}p{3cm}p{6cm}}
\hline
\multicolumn{2}{r}{T. Plan \href{https://rubinobs.atlassian.net/projects/LVV?selectedItem=com.atlassian.plugins.atlassian-connect-plugin:com.kanoah.test-manager__main-project-page\#\!/v2/testPlan/LVV-P83}{LVV-P83} owner:} &
\multicolumn{2}{l}{\textbf{ Yusra AlSayyad } }\\\hline
\multicolumn{2}{r}{T. Cycle \href{https://rubinobs.atlassian.net/projects/LVV?selectedItem=com.atlassian.plugins.atlassian-connect-plugin:com.kanoah.test-manager__main-project-page\#\!/testPlayer/LVV-R268}{LVV-R268} owner:} &
\multicolumn{2}{l}{\textbf{
Yusra AlSayyad }
} \\\hline
\textbf{Test Cases} & \textbf{Assigned to} & \textbf{Executed by} & \textbf{Additional Test Personnel} \\ \hline

\href{https://rubinobs.atlassian.net/projects/LVV?selectedItem=com.atlassian.plugins.atlassian-connect-plugin:com.kanoah.test-manager__main-project-page\#\!/v2/testCase/LVV-T362}{LVV-T362}
& {\small Yusra AlSayyad } & {\small Yusra AlSayyad } &
\begin{minipage}[]{6cm}
\smallskip
{\small  }
\medskip
\end{minipage}
\\ \hline

\href{https://rubinobs.atlassian.net/projects/LVV?selectedItem=com.atlassian.plugins.atlassian-connect-plugin:com.kanoah.test-manager__main-project-page\#\!/v2/testCase/LVV-T1601}{LVV-T1601}
& {\small Yusra AlSayyad } & {\small Yusra AlSayyad } &
\begin{minipage}[]{6cm}
\smallskip
{\small  }
\medskip
\end{minipage}
\\ \hline

\href{https://rubinobs.atlassian.net/projects/LVV?selectedItem=com.atlassian.plugins.atlassian-connect-plugin:com.kanoah.test-manager__main-project-page\#\!/v2/testCase/LVV-T363}{LVV-T363}
& {\small Yusra AlSayyad } & {\small Yusra AlSayyad } &
\begin{minipage}[]{6cm}
\smallskip
{\small  }
\medskip
\end{minipage}
\\ \hline
\end{longtable}
}

\newpage

\section{Test Campaign Overview}
\label{sect:overview}

\subsection{Summary}
\label{sect:summarytable}

{\small
\begin{longtable}{p{2cm}cp{2.3cm}p{8.6cm}p{2.3cm}}
\toprule
\multicolumn{2}{r}{ T. Plan \href{https://rubinobs.atlassian.net/projects/LVV?selectedItem=com.atlassian.plugins.atlassian-connect-plugin:com.kanoah.test-manager__main-project-page\#\!/v2/testPlan/LVV-P83}{LVV-P83}:} &
\multicolumn{2}{p{10.9cm}}{\textbf{ LDM-503-17a: Science Pipelines Final Release }} & Completed \\\hline
\multicolumn{2}{r}{ T. Cycle \href{https://rubinobs.atlassian.net/projects/LVV?selectedItem=com.atlassian.plugins.atlassian-connect-plugin:com.kanoah.test-manager__main-project-page\#\!/testPlayer/LVV-R268}{LVV-R268}:} &
\multicolumn{2}{p{10.9cm}}{\textbf{ LDM-503-17a: Science Pipelines Final Release }} & Done \\\hline
\textbf{Test Cases} &  \textbf{Ver.} & \textbf{Status} & \textbf{Comment} & \textbf{Issues} \\\toprule
\href{https://rubinobs.atlassian.net/projects/LVV?selectedItem=com.atlassian.plugins.atlassian-connect-plugin:com.kanoah.test-manager__main-project-page#!/v2/testCase/LVV-T362}{LVV-T362}
&
\\
 \hfill Execution & LVV-E3129
& Pass w/ Deviation &
\begin{minipage}[]{9cm}
\smallskip
None
\medskip
\end{minipage}
&
      \\\hline
  \href{https://rubinobs.atlassian.net/projects/LVV?selectedItem=com.atlassian.plugins.atlassian-connect-plugin:com.kanoah.test-manager__main-project-page#!/v2/testCase/LVV-T1601}{LVV-T1601}
&
\\
 \hfill Execution & LVV-E3130
& Pass &
\begin{minipage}[]{9cm}
\smallskip
None
\medskip
\end{minipage}
&
     \\\hline
  \href{https://rubinobs.atlassian.net/projects/LVV?selectedItem=com.atlassian.plugins.atlassian-connect-plugin:com.kanoah.test-manager__main-project-page#!/v2/testCase/LVV-T363}{LVV-T363}
&
\\
 \hfill Execution & LVV-E3131
& Pass &
\begin{minipage}[]{9cm}
\smallskip
None
\medskip
\end{minipage}
&
      \\\hline
     \caption{Test Campaign Summary}
\label{table:summary}
\end{longtable}
}

\subsection{Overall Assessment}
\label{sect:overallassessment}

All tests described in this plan were completed successfully. The
lsstinstall method automatically found and installed the appropriate
binaries of the LSST Science Pipelines v29.2.1 on the USDF developer
hardware.

\subsection{Recommended Improvements}
\label{sect:recommendations}

None

\newpage
\section{Detailed Test Results}
\label{sect:detailedtestresults}

\subsection{Test Cycle LVV-R268 }

Open test cycle {\it \href{https://rubinobs.atlassian.net/projects/LVV?selectedItem=com.atlassian.plugins.atlassian-connect-plugin:com.kanoah.test-manager__main-project-page\#\!/testPlayer/LVV-R268}{LDM-503-17a: Science Pipelines Final Release}} in Jira.

Test Cycle name: LDM-503-17a: Science Pipelines Final Release\\
Status: Done

This test cycle describes tests performed on the Science Pipelines Fall
2025 (v29.0.0) release, ensuring that the release is properly
identified, documented, distributed, installable and tested.

\subsubsection{Software Version/Baseline}
A web browser is required for inspecting release artifacts published to
the web.\\
\strut \\
Testing the software installation procedures, and demonstrating that the
release\textquotesingle s integration tests can be executed
successfully, require a supported operating system with the documented
prerequisites of the release installed. This will be carried out on the
''rubin-devl'' shared developer systems at the LSST Data Facility.\\
\strut \\
Science Pipelines prerequisites are documented
at~\href{https://pipelines.lsst.io/}{pipelines.lsst.io}; all of these
must be installed.\\
\strut \\
It is possible that the software release will involve a reorganization
of documentation describing prerequisites; in this case, the
documentation corresponding to the new release must be consulted.

\subsubsection{Configuration}
No specific configuration is required beyond the availability of test
systems with the prerequisite software, described above, installed.

\subsubsection{Test Cases in LVV-R268 Test Cycle}

\paragraph{ LVV-T362 - Installation of the LSST Science Pipelines Payloads }\mbox{}\\

Version \textbf{1.0(d)}.
Status \textbf{Approved}.
Open  \href{https://rubinobs.atlassian.net/projects/LVV?selectedItem=com.atlassian.plugins.atlassian-connect-plugin:com.kanoah.test-manager__main-project-page\#\!/v2/testCase/LVV-T362}{\textit{ LVV-T362 } }
test case in Jira.

This test will check that:

\begin{itemize}
\tightlist
\item
  The Alert Production Pipeline payload is available for installation
  from documented channels;
\item
  The Data Release Production Pipeline payload is available for
  installation from documented channels;
\item
  The Calibration Products Production Pipeline payload is available for
  installation from documented channels;
\item
  These payloads can be installed on systems at the LSST Data Facility
  following available documentation;
\item
  The installed pipeline payloads are capable of successfully executing
  basic integration tests.
\end{itemize}

Note that this test assumes packaging of the Science Pipelines software,
in which all the above payloads are represented by a single
``meta-package'', lsst\_distrib.

\textbf{ Preconditions}:\\ None


Execution status: {\bf Pass w/ Deviation }\\
Final comment:\\None



% Note Steps "Not Executed" and with No Result are not shown in this report if the flag is passed
Detailed steps results LVV-R268-LVV-E3129-1243141750:\\
  {

\begin{tabular}{p{4cm}p{12cm}}
\toprule
Step LVV-E3129-1 & Step Execution Status: \textbf{ Pass } \\ \hline
\end{tabular}
 Description \\
{\footnotesize
The LSST Science Pipelines, described by the lsst\_distrib meta-package,
should be installed following the documentation available at
https://pipelines.lsst.io/. The suggested Conda environment will be used
to ensure that a supported execution environment is available.

}
\hdashrule[0.5ex]{\textwidth}{1pt}{3mm}
  Test Data \\
 {\footnotesize
None

}
\hdashrule[0.5ex]{\textwidth}{1pt}{3mm}
  Expected Result \\
{\footnotesize
Detailed output will depend on the installation method chosen, but will
confirm the successful installation of the Science Pipelines.

}
\hdashrule[0.5ex]{\textwidth}{1pt}{3mm}
  Actual Result \\
{\footnotesize
The Science Pipelines were installed on the shared server sdfiana013 at
the USDF which is running 4.18.0-372.32.1.el8\_6.x86\_64, using the
procedure described in
https://pipelines.lsst.io/install/lsstinstall.html The rubin-env conda
environment and \hspace{0pt}Pipelines \hspace{0pt}binaries were
installed~\\
\strut \\

\begin{verbatim}
$ mkdir -p lsst_stack
$ cd lsst_stack
\end{verbatim}

\begin{verbatim}
$ curl -OL https://ls.st/lsstinstall
$ chmod u+x lsstinstall
$ ./lsstinstall -T v29_2_1
\end{verbatim}

\begin{verbatim}
$ source loadLSST.sh
\end{verbatim}

\begin{verbatim}
(lsst-scipipe-10.1.0)$ eups distrib install -t v29_2_1 lsst_distrib
(lsst-scipipe-10.1.0)$ curl -sSL https://raw.githubusercontent.com/lsst/shebangtron/main/shebangtron | python
\end{verbatim}

\hfill\break
\hfill\break
\hfill\break


}
  {

\begin{tabular}{p{4cm}p{12cm}}
\toprule
Step LVV-E3129-2 & Step Execution Status: \textbf{ Pass w/ Deviation } \\ \hline
\end{tabular}
 Description \\
{\footnotesize
The lsst\_distrib top-level metapackage will be enabled. Assuming that
the software has been installed at \$\{LSST\_DIR\}:\\
\strut \\
\strut ~ ~ ~ ~source \$\{LSST\_DIR\}/loadLSST.bash\\
\strut ~ ~ ~ ~setup lsst\_distrib

}
\hdashrule[0.5ex]{\textwidth}{1pt}{3mm}
  Test Data \\
 {\footnotesize
None

}
\hdashrule[0.5ex]{\textwidth}{1pt}{3mm}
  Expected Result \\
{\footnotesize
Nothing is printed. The command\\
\strut \\
\strut ~ ~eups list -s lsst\_distrib\\
\strut \\
may be used to confirm that the correct version of the codebase has been
installed.

}
\hdashrule[0.5ex]{\textwidth}{1pt}{3mm}
  Actual Result \\
{\footnotesize
\begin{verbatim}
The correct version was setup, but was not tagged v29_2_1. 

$ source loadLSST.bash
$ setup lsst_distrib
$ eups list -s lsst_distrib
   gc675d380bf+01ded468f7       current setup
\end{verbatim}


}
  {

\begin{tabular}{p{4cm}p{12cm}}
\toprule
Step LVV-E3129-3 & Step Execution Status: \textbf{ Pass } \\ \hline
\end{tabular}
 Description \\
{\footnotesize
The ``LSST Stack Demo'' package will be downloaded onto the test system
from https://github.com/lsst/pipelines\_check/releases. The version
corresponding to to the version of the Science Pipelines under test
should be chosen.

}
\hdashrule[0.5ex]{\textwidth}{1pt}{3mm}
  Test Data \\
 {\footnotesize
None

}
\hdashrule[0.5ex]{\textwidth}{1pt}{3mm}
  Expected Result \\
{\footnotesize
Depends on the tool selected by the user for downloading.

}
\hdashrule[0.5ex]{\textwidth}{1pt}{3mm}
  Actual Result \\
{\footnotesize
\begin{verbatim}
$ git clone  https://github.com/lsst/pipelines_check
$ cd pipelines_check
$ git checkout 25.2.1
HEAD is now at 6f05859 Merge pull request #55 from lsst/tickets/DM-49094
\end{verbatim}


}
  {

\begin{tabular}{p{4cm}p{12cm}}
\toprule
Step LVV-E3129-4 & Step Execution Status: \textbf{ Pass } \\ \hline
\end{tabular}
 Description \\
{\footnotesize
The stack demo package is uncompressed into a directory \$\{DEMO\_DIR\}.

}
\hdashrule[0.5ex]{\textwidth}{1pt}{3mm}
  Test Data \\
 {\footnotesize
None

}
\hdashrule[0.5ex]{\textwidth}{1pt}{3mm}
  Expected Result \\
{\footnotesize
Depends on options given to the tar command. Should confirm the
availability of the stack demo source.

}
\hdashrule[0.5ex]{\textwidth}{1pt}{3mm}
  Actual Result \\
{\footnotesize
The stack demo source code downloaded via git is available.~


}
  {

\begin{tabular}{p{4cm}p{12cm}}
\toprule
Step LVV-E3129-5 & Step Execution Status: \textbf{ Pass w/ Deviation } \\ \hline
\end{tabular}
 Description \\
{\footnotesize
The demo package will be executed by following the instructions in its
README file.~

}
\hdashrule[0.5ex]{\textwidth}{1pt}{3mm}
  Test Data \\
 {\footnotesize
None

}
\hdashrule[0.5ex]{\textwidth}{1pt}{3mm}
  Expected Result \\
{\footnotesize
Successful execution will result in the string ``Ok'' being returned.

}
\hdashrule[0.5ex]{\textwidth}{1pt}{3mm}
  Actual Result \\
{\footnotesize
\begin{verbatim}
$ setup -j -r . 
$ ./bin/run_demo.sh
<snip>
tests/test_butler.py ....                                                                                       [ 44%]
tests/test_validate_outputs.py .....                                                                            [100%]

================================================= 9 passed
\end{verbatim}


}
  % end if not not executed - no steps if not executed
\paragraph{ LVV-T1601 - Science Pipelines available on developer hardware }\mbox{}\\

Version \textbf{1.0(d)}.
Status \textbf{Approved}.
Open  \href{https://rubinobs.atlassian.net/projects/LVV?selectedItem=com.atlassian.plugins.atlassian-connect-plugin:com.kanoah.test-manager__main-project-page\#\!/v2/testCase/LVV-T1601}{\textit{ LVV-T1601 } }
test case in Jira.

This test will check that a given release of the LSST Science Pipelines
is available for use in a ``shared stack'' on developer infrastructure.

\textbf{ Preconditions}:\\ None


Execution status: {\bf Pass }\\
Final comment:\\None



% Note Steps "Not Executed" and with No Result are not shown in this report if the flag is passed
Detailed steps results LVV-R268-LVV-E3130-1243141751:\\
  {

\begin{tabular}{p{4cm}p{12cm}}
\toprule
Step LVV-E3130-1 & Step Execution Status: \textbf{ Pass } \\ \hline
\end{tabular}
 Description \\
{\footnotesize
Consult the LSST Developer Guide (http://developer.lsst.io/) to
establish:\\
\strut \\

\begin{itemize}
\tightlist
\item
  An appropriate hostname and login instructions for the shared
  developer infrastructure at the LSST Data Facility;
\item
  Instructions for initializing the shared stack on the developer
  infrastructure.
\end{itemize}

}
\hdashrule[0.5ex]{\textwidth}{1pt}{3mm}
  Test Data \\
 {\footnotesize
None

}
\hdashrule[0.5ex]{\textwidth}{1pt}{3mm}
  Expected Result \\
{\footnotesize
The Developer Guide clearly presents information about connecting to and
using shared infrastructure.

}
\hdashrule[0.5ex]{\textwidth}{1pt}{3mm}
  Actual Result \\
{\footnotesize
* Instructions for logging into the head nodes at the USDF are available
at https://developer.lsst.io/usdf/lsst-login.html\\
* Instructions for accessing the shared stack are available at
https://developer.lsst.io/usdf/stack.html


}
  {

\begin{tabular}{p{4cm}p{12cm}}
\toprule
Step LVV-E3130-2 & Step Execution Status: \textbf{ Pass } \\ \hline
\end{tabular}
 Description \\
{\footnotesize
Connect to the shared infrastructure following the Developer Guide
instructions.

}
\hdashrule[0.5ex]{\textwidth}{1pt}{3mm}
  Test Data \\
 {\footnotesize
None

}
\hdashrule[0.5ex]{\textwidth}{1pt}{3mm}
  Expected Result \\
{\footnotesize
A shell prompt on a shared machine.

}
\hdashrule[0.5ex]{\textwidth}{1pt}{3mm}
  Actual Result \\
{\footnotesize
login successful and a shell prompt is available.


}
  {

\begin{tabular}{p{4cm}p{12cm}}
\toprule
Step LVV-E3130-3 & Step Execution Status: \textbf{ Pass } \\ \hline
\end{tabular}
 Description \\
{\footnotesize
Initialize the LSST environment following the Developer Guide
instructions.

}
\hdashrule[0.5ex]{\textwidth}{1pt}{3mm}
  Test Data \\
 {\footnotesize
None

}
\hdashrule[0.5ex]{\textwidth}{1pt}{3mm}
  Expected Result \\
{\footnotesize
No errors are shown.

}
\hdashrule[0.5ex]{\textwidth}{1pt}{3mm}
  Actual Result \\
{\footnotesize
\begin{verbatim}
$ source /sdf/group/rubin/sw/w_latest/loadLSST.sh
\end{verbatim}


}
  {

\begin{tabular}{p{4cm}p{12cm}}
\toprule
Step LVV-E3130-4 & Step Execution Status: \textbf{ Pass } \\ \hline
\end{tabular}
 Description \\
{\footnotesize
List available software products using EUPS, and check that the release
under test is available.

}
\hdashrule[0.5ex]{\textwidth}{1pt}{3mm}
  Test Data \\
 {\footnotesize
None

}
\hdashrule[0.5ex]{\textwidth}{1pt}{3mm}
  Expected Result \\
{\footnotesize
The provided version number should be available in the shared stack.

}
\hdashrule[0.5ex]{\textwidth}{1pt}{3mm}
  Actual Result \\
{\footnotesize
An installation of v29 wasn\textquotesingle t available in
/sdf/group/rubin/sw/w\_latest/loadLSST.sh but it was available by\\
\strut \\

\begin{verbatim}
$ source /cvmfs/sw.lsst.eu/almalinux-x86_64/lsst_distrib/v29.2.1/loadLSST.bash

$ setup lsst_distrib
\end{verbatim}

which was the first location listed in the instructions on how to setup
the shared stack in~ https://developer.lsst.io/usdf/stack.html


}
  % end if not not executed - no steps if not executed
\paragraph{ LVV-T363 - Science Pipelines Release Documentation }\mbox{}\\

Version \textbf{1.0(d)}.
Status \textbf{Approved}.
Open  \href{https://rubinobs.atlassian.net/projects/LVV?selectedItem=com.atlassian.plugins.atlassian-connect-plugin:com.kanoah.test-manager__main-project-page\#\!/v2/testCase/LVV-T363}{\textit{ LVV-T363 } }
test case in Jira.

This test will check:

\begin{itemize}
\tightlist
\item
  That a particular Science Pipelines release is adequately described by
  documentation at the https://pipelines.lsst.io/ site;
\item
  That the Science Pipelines release is accompanied by a
  characterization report which describes its scientific performance.
\end{itemize}

\textbf{ Preconditions}:\\ None


Execution status: {\bf Pass }\\
Final comment:\\None



% Note Steps "Not Executed" and with No Result are not shown in this report if the flag is passed
Detailed steps results LVV-R268-LVV-E3131-1243141752:\\
  {

\begin{tabular}{p{4cm}p{12cm}}
\toprule
Step LVV-E3131-1 & Step Execution Status: \textbf{ Pass } \\ \hline
\end{tabular}
 Description \\
{\footnotesize
Load the Science Pipelines website at https://pipelines.lsst.io/.

}
\hdashrule[0.5ex]{\textwidth}{1pt}{3mm}
  Test Data \\
 {\footnotesize
None

}
\hdashrule[0.5ex]{\textwidth}{1pt}{3mm}
  Expected Result \\
{\footnotesize
The website is displayed.

}
\hdashrule[0.5ex]{\textwidth}{1pt}{3mm}
  Actual Result \\
{\footnotesize
The website https://pipelines.lsst.io is displayed


}
  {

\begin{tabular}{p{4cm}p{12cm}}
\toprule
Step LVV-E3131-2 & Step Execution Status: \textbf{ Pass } \\ \hline
\end{tabular}
 Description \\
{\footnotesize
Identify documentation for the release under test. This should be
clearly labelled on the documentation site.\\
\strut \\
If the latest release is being tested, the default page loaded when
visiting https://pipelines.lsst.io/ should be the documentation
required.\\
\strut \\
If this test is for another release, the site should present clear
instructions for changing the edition (or version) of the documentation
being examined, and documentation for the release under test should be
available.

}
\hdashrule[0.5ex]{\textwidth}{1pt}{3mm}
  Test Data \\
 {\footnotesize
None

}
\hdashrule[0.5ex]{\textwidth}{1pt}{3mm}
  Expected Result \\
{\footnotesize
The documentation for the release under test is displayed.

}
\hdashrule[0.5ex]{\textwidth}{1pt}{3mm}
  Actual Result \\
{\footnotesize
https://pipelines.lsst.io is for the release under test and clearly
states: \textquotesingle\textquotesingle This documentation covers
version \textbf{v29\_2\_1\textquotesingle\textquotesingle~}\\
It was also simple to navigate to "Other Versions", and click on
v29\_2\_1 and land on the permanent page address for this release:
https://pipelines.lsst.io/v/v25\_2\_1/index.html


}
  {

\begin{tabular}{p{4cm}p{12cm}}
\toprule
Step LVV-E3131-3 & Step Execution Status: \textbf{ Pass } \\ \hline
\end{tabular}
 Description \\
{\footnotesize
Inspect the documentation to ensure that it refers to the release under
test, and that it provides:

\begin{itemize}
\tightlist
\item
  Release notes, describing changes in this release relative to the
  previous;
\item
  Installation instructions, together with a list of supported platforms
  and prerequisites;
\item
  Getting started information.
\end{itemize}

}
\hdashrule[0.5ex]{\textwidth}{1pt}{3mm}
  Test Data \\
 {\footnotesize
None

}
\hdashrule[0.5ex]{\textwidth}{1pt}{3mm}
  Expected Result \\
{\footnotesize
The user is satisfied that the required information is available.

}
\hdashrule[0.5ex]{\textwidth}{1pt}{3mm}
  Actual Result \\
{\footnotesize
The documentation for the following are all available on the front page
or linked from the front page:

\begin{itemize}
\tightlist
\item
  https://pipelines.lsst.io/v/v29\_2\_1/releases/v29\_2\_0.html\#release-latest
\item
  Release notes:
  https://pipelines.lsst.io/v/v29\_2\_1/releases/v29\_2\_0.html\#release-latest
  which links to the corresponding major release:
  https://pipelines.lsst.io/v/v29\_2\_1/releases/v29\_0\_0.html\#release-v29-0-0
  which described most of the changes.~
\item
  Installation instructions:
  https://pipelines.lsst.io/v/v29\_2\_1/install/index.html
\item
  Getting started information:
  https://pipelines.lsst.io/v/v29\_2\_1/index.html\#getting-started
\end{itemize}


}
  {

\begin{tabular}{p{4cm}p{12cm}}
\toprule
Step LVV-E3131-4 & Step Execution Status: \textbf{ Pass } \\ \hline
\end{tabular}
 Description \\
{\footnotesize
Locate the Characterization Metric Report corresponding to this release.
It should be linked from the main release documentation.

}
\hdashrule[0.5ex]{\textwidth}{1pt}{3mm}
  Test Data \\
 {\footnotesize
None

}
\hdashrule[0.5ex]{\textwidth}{1pt}{3mm}
  Expected Result \\
{\footnotesize
The user is satisfied that the report is available.

}
\hdashrule[0.5ex]{\textwidth}{1pt}{3mm}
  Actual Result \\
{\footnotesize
The characterization report (dmtr-461.lsst.io) is linked from the
release notes at
https://pipelines.lsst.io/v/v29\_2\_1/releases/v29\_2\_0.html\#release-latest\\
\strut \\


}
  {

\begin{tabular}{p{4cm}p{12cm}}
\toprule
Step LVV-E3131-5 & Step Execution Status: \textbf{ Pass } \\ \hline
\end{tabular}
 Description \\
{\footnotesize
Verify that the characterization metric report describes the scientific
performance of the release in terms of a selection of performance
metrics drawn from high-level requirements documentation (the Science
Requirements Document, LPM-17; the LSST System Requirements, LSE-29;
and/or the Observatory System Specifications, LSE-30).

}
\hdashrule[0.5ex]{\textwidth}{1pt}{3mm}
  Test Data \\
 {\footnotesize
None

}
\hdashrule[0.5ex]{\textwidth}{1pt}{3mm}
  Expected Result \\
{\footnotesize
Metric values describing the performance of the release, for example as
computed by validate\_drp, are described in the report.

}
\hdashrule[0.5ex]{\textwidth}{1pt}{3mm}
  Actual Result \\
{\footnotesize
Metrics computed by faro\textquotesingle s successor, analysis\_tools,
are clearly described in the report and compared with the SRD thresholds
and the values in the previous release v28. It shows computed metrics on
precursor data as with prior characterization reports compared with new
LSSTComCam~ DP1 data.


}
  % end if not not executed - no steps if not executed
  %end of the if with theo test_items in testcycles_map[cyclie.id]


\input{appendix.tex}
\end{document}
